\documentclass{tufte-handout}

\usepackage{xcolor}

% set image attributes:
\usepackage{graphicx}
\graphicspath{ {images/} }

% set hyperlink attributes
\hypersetup{colorlinks}

% create environment for bottom paragraph:
\newenvironment{bottompar}{\par\vspace*{\fill}}{\clearpage}

\usepackage{enumerate}

% set table attributes
\usepackage{tabu}
\usepackage{booktabs}

% ============================================================

% define the title
\title{SOC 4015/5050: PS-06 - Multivariate Regression}
\author{Christopher Prener, Ph.D.}
\date{Fall 2018}

% ============================================================

\begin{document}

% ============================================================

\maketitle % generates the title

% ============================================================

\vspace{5mm}
\section{Directions}
Please complete all steps below. All work should be uploaded to your GitHub assignment repository by 4:15pm on Monday, November 3\textsuperscript{rd}, 2018. All data can be obtained from the \texttt{testDriveR} package's \texttt{gss14} data set.

\vspace{5mm}
\section{Analysis Development}
Using RStudio and your operating system's file manager, create an R Project in the \textit{existing} directory in your assignments repository named \texttt{PS-06}. Add a \texttt{README.md} file, notebook, and all necessary folders before beginning.\sidenote{This initial section follows the project workflow that is available in the \texttt{lecture-03} repo!}

\vspace{5mm}
\section{Part 1: Data Preparation}
\begin{enumerate}
\item Using the data table \texttt{gss14} in the \texttt{testDriveR} package, create a new data frame that has \textit{only} the following data:
\begin{verbatim}
> gssClean
# A tibble: 2,538 x 8
      id hrsWork white black otherRace female fullTime incomeCat
   <int>   <int> <lgl> <lgl> <lgl>     <lgl>  <lgl>        <dbl>
 1     1      60 TRUE  FALSE FALSE     FALSE  TRUE            21
 2     2      40 TRUE  FALSE FALSE     TRUE   TRUE            25
 3     3      NA TRUE  FALSE FALSE     FALSE  FALSE           18
 4     4      20 TRUE  FALSE FALSE     TRUE   FALSE           25
 5     5      NA TRUE  FALSE FALSE     TRUE   FALSE           NA
 6     6      60 TRUE  FALSE FALSE     TRUE   TRUE            25
 7     7      NA TRUE  FALSE FALSE     FALSE  NA              NA
 8     8      40 TRUE  FALSE FALSE     FALSE  TRUE            21
 9     9      NA TRUE  FALSE FALSE     TRUE   FALSE           11
10    10      55 FALSE FALSE TRUE      TRUE   TRUE            22
# ... with 2,528 more rows
\end{verbatim}

Store these cleaned data in your \texttt{data/} sub-directory as a \texttt{.csv} file.
\end{enumerate}

\vspace{5mm}
\section{Part 2: Descriptive Statistics and Assumptions}
Using the GSS data created above in Part 1, answer the following questions. 
\begin{enumerate}
\setcounter{enumi}{1}
\item Report the \textit{appropriate} descriptive statistics for \textit{all} of the variables displayed in the output included with Part 1. Also create a formatted descriptive statistics table to include with your assignment submission. Store the output in your \texttt{results/} sub-directory.\sidenote{This output should be left as a \texttt{.html} file - it does not need to be reformatted into Microsoft Word.}
\item Conduct a full set of normality tests on the variables \texttt{hrsWork} and \texttt{incomeCat} and report your findings.\sidenote{For the purposes of this assignment, we are going to treat \texttt{incomeCat} as a continuous variable.}
\item Create a correlation table to identify any possible issues with regression assumptions.
\item Summarize your assessment of how these data meet the assumptions of linear regression.
\end{enumerate}

\vspace{5mm}
\section{Part 3: Model}
Using the GSS data created above in Part 1, answer the following questions.
\begin{enumerate}
\setcounter{enumi}{5}
\item Construct a hypothesis and null hypothesis for the relationship between number of hours worked (\texttt{hoursWork}) and income (\texttt{incomeCat}), accounting for the other factors included in your data set.
\item Construct a dissemination ready plot of the relationship between hours worked (\texttt{hoursWork}) and income (\texttt{incomeCat}). 
\item Construct a regression equation modeling how income, accounting for race, gender, and whether or not someone works full time, affects \texttt{hoursWork} using \LaTeX{}\ syntax. 
\item Execute a main effects model (model 1) of the effect of income on hours worked (\texttt{hoursWork}) (\texttt{incomeCat}).
\item Execute a full model  (model 2) with all of your control variables.
\item Provide a written summary of the findings of both of your models, including interpretations of the betas and appropriate measures of model fit.
\end{enumerate}

\vspace{5mm}
\section{Part 4: Post-Hoc Assumptions Checks}
Using the GSS data created above in Part 1, answer the following questions.
\begin{enumerate}
\setcounter{enumi}{13}
\item Using the skills covered in Lecture 14, \textit{fully} check the assumptions and model fit of your second model.
\item Provide a written summary of the findings of your assumption checks.
\end{enumerate}

\vspace{5mm}
\section{Part 5: Final Model}
Using the GSS data created above in Part 1, answer the following questions.
\begin{enumerate}
\setcounter{enumi}{15}
\item Fit another model (model 3) that properly accounts for any issues discovered in Part 4. 
\item Provide a written summary of how re-fitting the model has changed its conclusions. Is model 2 or model 3 a better model overall?
\end{enumerate}

% ============================================================
\end{document}